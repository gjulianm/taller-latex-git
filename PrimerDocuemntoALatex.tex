% vim:ts=2:sw=2
%
\pdfoutput=1
\ifx\synctex\undefined\else\synctex=1\fi
%
\documentclass{article}

\usepackage{amsmath}
\usepackage{amssymb}
\usepackage{amsthm}

\usepackage{color,xspace,hyperref}
\usepackage{float}
\usepackage{caption}
\usepackage[spanish, es-tabla,es-noquoting,es-noshorthands]{babel}


\author{}
\date{}

\title{Borrador}

\begin{document}
\maketitle


\begin{abstract}

\end{abstract}

\section{Fórmulas famosas}

\subsection{Suma de los $n$ primeros enteros}

% Escribir la serie y la ecuación

\subsection{Suma de los cuadrados de los $n$ primeros enteros}

% Escribir la serie y la ecuación

\subsection{Teorema fundamental del cálculo}

% Escribir el teorema fundamental del cálculo

\subsection{Logaritmo de un producto}

\begin{equation}[\log(a*b)=\log(a)+\log(b)\end{equation}]

$\log a·b = \log a + \log b\quad \forall a,b > 0$\label{form_log}

\subsection{Identidades notables}

% Escribir una enumeracion con las 3 igualdades notarias (a+b)^2, (a-b)^2, (a+b)(a-b)
\begin{enumerate}
\item $ \left(a+b\right)^2 = a^2 + 2ab + b^2 $
\item $ \left(a-b\right)^2 = a^2 - 2ab + b^2 $
\item $ \left(a+b\right)\left(a-b\right) = a^2-b^2 $
\end{enumerate}

\section{Desarrollo de Taylor}

\subsection{Definición general}

% Escribir la fórmula general del desarrollo de taylor

\subsection{Casos concreto}

Desarrollos de taylor de las funciones:

\begin{itemize}
\item $f(x)=e^x$

% Escribir desarrollo de Taylor de esta función

\item $f(x) = \sin (x)$

%Escribir desarrollo de Taylor de esta función

\end{itemize}

\section{Matrices}
\[ \begin{pmatrix}
   0 & 2 & 0 & 0 & 0 & 0 & 0 \\
   0 & 0 & 5 & 0 & 0 & 0 & 1 \\
   0 & 0 & 0 & 0 & 4 & 2 & 0 \\
   0 & 4 & 0 & 0 & 0 & 0 & 0 \\
   0 & 0 & 3 & 0 & 0 & 0 & 1
\end{pmatrix} \]

% Escribir la matriz
% 0 2 0 0 0 0 0
% 0 0 5 0 0 0 1
% 0 0 0 0 4 2 0
% 0 4 0 0 0 0 0
% 0 0 3 0 0 0 1

\subsection{Traspuesta de una matriz}

% Escribir la traspuesta de la matriz anterior
\[\begin{pmatrix}
	0 & 0 & 0 & 0 & 0 \\
	2 & 0 & 0 & 4 & 0 \\
	0 & 5 & 0 & 0 & 3 \\
	0 & 0 & 0 & 0 & 0 \\
	0 & 0 & 4 & 0 & 0 \\
	0 & 0 & 2 & 0 & 0 \\
	0 & 1 & 0 & 0 & 1
\end{pmatrix}\]

\section{Tablas}

% Escribir una tabla de 4 columnas y 5 filas rellena de 0s



\end{document}


